\documentclass{beamer}

\usepackage[utf8]{inputenc}
\usepackage{tikz}
\usepackage{hyperref}
\usepackage{listings}
\usepackage{float}
\usepackage{multicol}

\graphicspath{{./images/}}

\usetheme{Madrid}

\title{The Legend of Link}

\author{David Antuña \and Jaime Bas \and Irene González \and José Luis Moreno}

\institute[UCM]
{
  DVI\\
  Universidad Complutense de Madrid
}

\date{Curso 2017-18}

\subject{Making a game with Quintus}

\pgfdeclareimage[height=0.5cm]{university-logo}{images_ignore/university-logo-filename}
\logo{\pgfuseimage{university-logo}}

%%%%% Listings JS
\lstdefinelanguage{JavaScript}{
  keywords={typeof, new, true, false, catch, function, return, null, catch, switch, var, if, in, while, do, else, case, break},
  keywordstyle=\color{blue}\bfseries,
  ndkeywords={class, export, boolean, throw, implements, import, this},
  ndkeywordstyle=\color{red}\bfseries,
  identifierstyle=\color{black},
  sensitive=false,
  comment=[l]{//},
  morecomment=[s]{/*}{*/},
  commentstyle=\color{purple}\ttfamily,
  stringstyle=\color{red}\ttfamily,
  morestring=[b]',
  morestring=[b]"
}

\AtBeginSection[]
{
  \begin{frame}<beamer>{Contenidos}
    \begin{multicols}{2}
      \tableofcontents[currentsection]
    \end{multicols}
  \end{frame}
}

\begin{document}

\begin{frame}
	\titlepage
\end{frame}

\begin{frame}{Contenidos}
	\begin{multicols}{2}
		\tableofcontents
	\end{multicols}
\end{frame}


% Section and subsections will appear in the presentation overview
% and table of contents.
\section{Introducción al juego}

\begin{frame}{Introducción al juego}
	\begin{itemize}
		\item Basado en la saga The Legend of Zelda (Nintendo, 1986).
		\item Assets de Four Swords y Minish Cap.
		\item Simplificado, centrado en combates en una mazmorra sencilla.
		\item Victoria: derrotar al jefe final.
		\item Derrota: superado límite de impactos.
	\end{itemize}
\end{frame}

%%%%%%%%%%%%%%%%%%%%%%%%%%%%%%%%%%%%
\section{Mecánicas}

\begin{frame}{Mecánicas}
	\begin{itemize}
		\item Libertad de movimiento.
		\item Combate.
		\item Salud.
		\item Interfaz.
		\item Tracking.
		\item Puzzles.
		\item Interacción con el escenario.
	\end{itemize}
\end{frame}

%%%%%%%%%%%%%%%%%%%%%%%%%%%%%%%%%%%%
\section{Arquitectura de las clases}

\subsection{Personajes y objetos inanimados}

\begin{frame}{Personajes y objetos inanimados}
	\begin{itemize}
		\item Link y enemigos.
		\item Basados en Sprite.
		\item Múltiples parámetros propios de cada personaje.
		\item Cofres y rupias.
	\end{itemize}
\end{frame}

\subsection{Escenario y pantallas}

\begin{frame}{Escenario y pantallas}
	\begin{itemize}
		\item Mapa único con zonas de transición.
		\item Varias habitaciones.
		\item Menú principal.
		\item Pantalla de créditos.
	\end{itemize}
\end{frame}

%%%%%%%%%%%%%%%%%%%%%%%%%%%%%%%%%%%%
\section{Personajes}

\subsection{Personaje principal}

\begin{frame}{Personaje principal}
	\begin{itemize}
		\item Mecánicas.
		      \begin{itemize}
			      \item Interactuar con objetos.
			      \item Atacar enemigos.
			      \item Recibir daño.
			      \item Movimiento.
		      \end{itemize}
	\end{itemize}
\end{frame}

\subsection{Enemigos}

\begin{frame}{Enemigos}
	\begin{itemize}
		\item Darknut.
		\item Shadow Link.
		\item Varias habitaciones.
		      \begin{itemize}
			      \item Persigue al jugador.
			      \item Ataca.
			      \item Recibe daño.
		      \end{itemize}
	\end{itemize}
\end{frame}

%%%%%%%%%%%%%%%%%%%%%%%%%%%%%%%%%%%%
\section{Objetos}

\begin{frame}{Objetos}
	\begin{itemize}
		\item Cofres.
		      \begin{itemize}
			      \item Contienen una rupia.
			      \item Se abren cuando el jugador interactúa con ellos estando al lado.
		      \end{itemize}
		      \medskip
		\item Rupias.
		      \begin{itemize}
			      \item Aparecen al abrir un cofre.
			      \item Están animadas con tween.
		      \end{itemize}
	\end{itemize}
\end{frame}

%%%%%%%%%%%%%%%%%%%%%%%%%%%%%%%%%%%%
\section{aiTrack}

\subsection{¿Qué es?}

\begin{frame}{¿Qué es?}
	\begin{itemize}
		\item Nuevo componente 2d.
		\item Otorga a un sprite capacidad para:
		      \begin{itemize}
			      \item Localizar.
			      \item Seguir.
			      \item Atacar.
		      \end{itemize}
		\item Basado en cuadrícula $\rightarrow$ Solo 2d con vista cenital.
	\end{itemize}
\end{frame}

\subsection{Propiedades}

\begin{frame}{Propiedades}
	\begin{itemize}
		\item \textit{view\_range} y \textit{attack\_range}: Rango de activación.
		\item \textit{tile\_size}: Tamaño de un tile de la cuadrídula.
		\item \textit{vfactor}: Multiplicador de velocidad\\
		      \hspace{5mm}v = tile\_size * vfactor
		\item \textit{attacking} y \textit{tracking}: Indicadores de acción en curso
		\item \textit{reloadSpeed}: Tiempo de enfriamiento de un ataque
		\item \textit{reload}: Enfriamiento actual
		\item \textit{track\_class}: Lista de clases con las que el sprite interactuará
	\end{itemize}
\end{frame}

\subsection{Funcionamiento}

\begin{frame}{Funcionamiento}
	En cada iteración del bucle de juego el modulo Scenes activará la señal
	\textbf{range.check} para todos los componentes de la escena.

	\medskip

	El componente escucha esta señal y comprueba la distancia entre la entidad y
	los elementos de la escena cuya clase esté en la lista.

	\medskip

	Hay tres posibilidades:
	\begin{itemize}
		\item No está en rango $\rightarrow$ No ocurre nada
		\item Está en rango de ataque $\rightarrow$ Activa la señal
		      \textbf{range.attack}
		\item Está en rango de movimiento $\rightarrow$ Activa la señal
		      \textbf{range.view}
	\end{itemize}
\end{frame}

\subsection{Triggers}

\begin{frame}{range.attack}
	Esta señal activa la función attack del componente.

	\medskip

	Si reload es menor o igual que 0.
	\begin{itemize}
		\item Activa el atributo attacking
		\item Resetea el tiempo de enfriamiento
		\item Invoca la función hit y le pasa el atributo damage de la entidad
	\end{itemize}
\end{frame}

\begin{frame}{range.view}
	Esta señal activa la función move del componente.

	\medskip

	Esta función trata de equiparar el componente x e y de la entidad a los del
	elemento que esta siguiendo.

	\begin{center}
		vx = $\pm$ tile\_size * vfactor\\
		vy = $\pm$ tile\_size * vfactor
	\end{center}
\end{frame}

%%%%%%%%%%%%%%%%%%%%%%%%%%%%%%%%%%%%
\section{Componente stepControls}

\subsection{Problema}

\begin{frame}{Problema}
	Para que el sprite se pueda mover en las 4 direcciones usamos el componente
	de Quintus stepControls pero genera clipping al combinarse con nuestro aiTrack.

	\medskip

	El componente comprueba si la entidad colisiona y de hacerlo la devuelve al
	origen, el problema es que los enemigos te persiguen y este componente impide
	que el sprite escape.
\end{frame}

\begin{frame}{Código}
	\lstinputlisting[language=JavaScript]{code/stepProblem.js}
\end{frame}

\subsection{Solución}

\begin{frame}{Solución}
	Para solucionar este problema nos aprovechamos de un atributo de los
	sprites en los que usamos este componente, todos tienen el atributo
	\textbf{direction}.

	\medskip

	Gracias a dicho atributo sabemos en que dirección se esta moviendo por lo que
	podemos modificar el código para que se resetee la posicion solo si se mueve
	hacia el objeto.
\end{frame}

\begin{frame}{Código}
	\lstinputlisting[language=JavaScript]{code/stepSol.js}
\end{frame}

%%%%%%%%%%%%%%%%%%%%%%%%%%%%%%%%%%%%
\section{Puzzles}

\begin{frame}{Puzzles}
	\begin{itemize}
		\item Propiedades.
		      \begin{itemize}
			      \item Sprite $\rightarrow$ \textit{activated}.
			      \item Stage $\rightarrow$ \textit{label}.
		      \end{itemize}
		\item Funcionamiento.
	\end{itemize}
\end{frame}

%%%%%%%%%%%%%%%%%%%%%%%%%%%%%%%%%%%%
\section{Postmortem}

\subsection{¿Qué nos ha ido bien?}

\begin{frame}{¿Qué nos ha ido bien?}
	\begin{itemize}
		\item Idea muy clara desde el principio.
		\item Tareas bien repartidas y trabajo simultáneo
	\end{itemize}
\end{frame}

\subsection{¿Qué nos ha ido mal o se podría mejorar?}

\begin{frame}{¿Qué nos ha ido mal o se podría mejorar?}
	\begin{itemize}
		\item Complicaciones con Tiled.
		\item Implementación del cambio de habitación.
		\item Proyecto demasiado ambicioso.
	\end{itemize}
\end{frame}
\end{document}
